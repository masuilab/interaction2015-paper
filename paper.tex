%
% 「降臨鉄道」インタラクション2015ポスタ
%
% 山田尚昭 and 増井俊之
% 2014/12/13 17:40:18
%
% 以下を eval-region して句読点設定
% (replace-string "。" ".")
% (replace-string "、" ",")
%

\documentclass[submit,techreq]{ipsj}

\usepackage[dvipdfmx]{graphicx}

\usepackage{latexsym}
\usepackage{here} % [H]とするとその場所に配置される

% ???
\def\Underline{\setbox0\hbox\bgroup\let\\\endUnderline}
\def\endUnderline{\vphantom{y}\egroup\smash{\underline{\box0}}\\}
\def\|{\verb|}

% インタラクション特有の設定.印刷工程で柱・ノンブルの埋め込みを行う.
\makeatletter
\pagestyle{empty}
\def\@oddhead{}
\def\@evenhead{}
\def\ps@IPSJTITLEheadings{}
\makeatother

\begin{document}

\title{降臨鉄道: 模型モノレールを利用した遠隔通信}
\etitle{Camera on Rails: Telecommunication with Model Monorail Trains}

\affiliate{KU}{慶應義塾大学 環境情報学部\\
Faculty of Environment and Infomation Studies, Keio University}

\author{山田 尚昭}{Naoaki Yamada}{KU}
\author{増井 俊之}{Toshiyuki Masui}{KU}

\begin{abstract}
カメラを登載した模型モノレールをオフィスの天井で走らせることによって,
どこからでもオフィスの様子を覗いたりオフィス内の人間と会話したりできる遠隔通信システムを作成した.
車輪で移動可能なロボットを利用することによって
遠隔地のユーザが会議や学会に参加する試みが近年盛んになっているが,
混雑した環境ではロボットが自由に移動できないため実運用が難しいことが多い.
邪魔物が無い天井に装着したレール上を自由に移動できるモノレールを利用することにより,
実用的な遠隔コミュニケーションシステムが実現できた.
% 本システムの利用により,
% 卒業したメンバが部室を訪れる「OB降臨」機能を手軽に実現することができる.
\end{abstract}

\begin{eabstract}
We developed the ``Camera on Rails'' telecommunication system with
which a user can communicate with other people in a
distance office through a camera on a model monorail train running on the ceiling
of the office. Using our system, the user can monitor the current status
of the office by running the train on the ceiling and have
conversation with the people in the office.
% from arbitrary angles.
\end{eabstract}

\maketitle

\section{はじめに}

遠隔地の人とあたかも同じ場所にいるかのような感覚を強化する
テレプレゼンスシステムの研究が盛んである.
今日ではビデオ会議システムや遠隔操作可能な人間を模したロボットなどが普及しつつある.

以前から研究室において卓上を動き回ることのできるロボットを介した遠隔コミュニケーション
支援システムを製作して運用していたが(図\ref{Guntank})\cite{Hirota:Korin},
動ける範囲に制限があったり障害物に阻まれてカメラからの映像が見られなかったり
することが多かった.

何にも邪魔をされることのない部屋の天井を移動する遠隔情報共有システム
降臨鉄道を提案する.

\section{関連研究}

Jouppiのシステム\cite{Jouppi:2002:FST:587078.587128}では自由に動き回れる
ロボットで遠隔地を訪れられる.大掛かりな操作室が必要だが,現地での活動能力は高い.

Double RoboticsのDouble\footnote{
  \textsf{http://www.doublerobotics.com}
}(図\ref{double})
やSuitable Technologiesのbeam pro\footnote{
  \textsf{http://https://www.suitabletech.com/beampro/}
}(図\ref{beampro})
は遠隔地から操作可能な人間を模したテレプレゼンスロボットである.Doubleやbeam pro
のような人間の分身を模したロボットは,卓上に置かれた小型のロボットに比べて移動
できる範囲は大きいが,他人の邪魔になることがある.

\begin{figure}[H]
\centerline{\includegraphics[width=50mm]{figures/1e8781bb2a5b28c8e06906d226c7505a.png}}
\caption{ガンタンク.}
\label{Guntank}
\end{figure}


\begin{figure}[H]
\centerline{\includegraphics[width=50mm]{figures/b74f4564d4b38d12e48fcf80fef96def.png}}
\caption{WISS2014で質疑応答に利用されたDouble.}
\label{double}
\end{figure}
% http://engineer.typemag.jp/article/yuki-igarashi41

\begin{figure}[H]
\centerline{\includegraphics[width=50mm]{figures/2c092d5d4467d5b2572acef0c95b22ff.png}}
\caption{BeamPro.}
\label{beampro}
\end{figure}

%\footnote{
%  \textsf{https://www.suitabletech.com/beam/}
%}

\section{降臨鉄道}

ユーザはWebページにアクセスすることで降臨鉄道が設置された遠隔地のリアルタイム
映像を見たり,降臨鉄道を操作して移動させることができる(図\ref{monorail}).

\begin{figure}[H]
\begin{center}
\includegraphics[width=50mm]{figures/image.png}
\end{center}
\caption{天井を走る降臨鉄道.}
\label{monorail}
\end{figure}


降臨鉄道システムは,Androidスマートフォンが取り付けられた懸垂式モノレール型
鉄道模型の降臨鉄道とWebサーバ,linda-serverから構成される.

降臨鉄道にはAndroidスマートフォンが搭載されており,サーバからの命令を受け取り
DCモータを制御する.DCモータの制御は440Hzの正弦波のMP3音声ファイルを再生し,
イヤホン端子から擬似的な交流電源を得てリレーを駆動させることによって行っている.
電源はUSB接続のモバイルバッテリーを使用している。

動画の配信にはAndroidアプリケーションIP Webcam\footnote{
  \textsf{https://play.google.com/store/apps/details?id=com.pas.webcam}
}
を使用し,MotionJPEG形式の映像をHTTPで配信している.

WebサーバにはAndroid用のページとクライアント用のページが
用意されている(図\ref{browser}).

\begin{figure}[H]
\begin{center}
\includegraphics[width=70mm]{figures/korin.png}
\end{center}
\caption{クライアントのWebサイト. モノレールからの映像が表示されている.}
\label{browser}
\end{figure}

クライアント用ページでは降臨鉄道のカメラからの映像の表示と移動の命令をし,
Android用ページでは移動の命令に従って音声ファイルを再生している.

操作は並列計算プリミティブLinda\cite{Carriero:1989:LC:63334.63337}
をWebサーバ上に実装したlinda-server\footnote{
  \textsf{https://github.com/node-linda/linda}
}
を用いて実装している.

\section{まとめと展望}

今回のデモで紹介したシステムは現在,研究室内で実際の運用を通じて実験中である.
降臨鉄道は遠隔コミュニケーションシステムとして利用するのはもちろんのこと、
多くの人が集まるイベント会場などで遠隔イベント参加システムとして利用できるだろう.

本実装では使用している鉄道模型が1つしかないため、複数の人が参加することが
できない,レールに沿って移動するため任意の場所に移動できないということがあるが,
これらは今後の課題としたい.また,自動でバッテリーの充電を行う「駅」を作ること
で運用する上での負担を減らしていきたい.

将来的には目的に応じたロボットを複数個用意し,世代間を超えてより活発な
研究活動ができるようにしたいと考えている.

{\scriptsize
\bibliographystyle{ipsjsort}
\bibliography{paper}
}

\end{document}
